\documentclass[11pt]{article}

% Any additional packages needed should be included after jmlr2e.
% Note that jmlr2e.sty includes epsfig, amssymb, natbib and graphicx,
% and defines many common macros, such as 'proof' and 'example'.
%
% It also sets the bibliographystyle to plainnat; for more information on
% natbib citation styles, see the natbib documentation, a copy of which
% is archived at http://www.jmlr.org/format/natbib.pdf

% \usepackage[preprint]{jmlr2e}
\usepackage[margin=1in]{geometry}
\usepackage{amssymb}
\usepackage{natbib}
\usepackage{color}
% \usepackage{float}
% \usepackage{hyperref}
\usepackage[hidelinks,colorlinks=true,citecolor=blue]{hyperref}
\usepackage{amsfonts}       
\usepackage{url}
\usepackage{amsmath}
\usepackage{amssymb}
\usepackage{amsthm}
\usepackage{wrapfig}
\usepackage{mathtools}
\usepackage{tikz}
\usepackage{subfig}
\usepackage{algorithmic}
\usepackage{subfig}
\usepackage{footmisc}
\usepackage{bibentry}
\usepackage{mathtools}
\DeclarePairedDelimiter{\ceil}{\lceil}{\rceil}
\usepackage{physics}
\usepackage{mathrsfs}
\mathtoolsset{showonlyrefs}
\usepackage{etoolbox}
\usepackage{makecell}
\usepackage[ruled]{algorithm2e}
\DontPrintSemicolon
\usepackage{enumerate}
\usepackage{booktabs}
\usepackage{listings}

\newcommand{\sz}[1]{({\color{blue} {SZ: #1}})}
\newcommand{\rl}[1]{({\color{green} {RL: #1}})}
\newcommand{\rt}[1]{({\color{red} {RT: #1}})}
\newcommand{\fS}{\mathcal{S}}
\newcommand{\fA}{\mathcal{A}}
\newcommand{\fY}{\mathcal{Y}}
\newcommand{\fX}{\mathcal{X}}
\newcommand{\fN}{\mathcal{N}}
\newcommand{\mY}{\mathbb{Y}}
\newcommand{\mP}{\mathbb{P}}
\newcommand{\fB}{\mathcal{B}}
\newcommand{\mI}{\mathbb{I}}
\newcommand{\fZ}{\mathcal{Z}}
\newcommand{\mZ}{\mathbb{Z}}
\newcommand{\fW}{\mathcal{W}}
\newcommand{\fF}{\mathcal{F}}
\newcommand{\fO}{\mathcal{O}}
\newcommand{\fH}{\mathcal{H}}
\newcommand{\mH}{\mathbb{H}}
\newcommand{\fU}{\mathcal{U}}
\newcommand{\R}{\mathbb{R}}
\newcommand{\Q}{\mathbb{Q}}
\newcommand{\E}{\mathbb{E}}
\newcommand{\nsa}{{|\fS \times \fA|}}
\newcommand{\ns}{{|\fS|}}
\newcommand{\na}{{|\fA|}}
\newcommand{\ny}{{|\fY|}}
\newcommand{\tb}[1]{{\textbf{#1}}}
\DeclarePairedDelimiter\floor{\lfloor}{\rfloor}

\begin{document}

\title{CS 4501: Optimization - Assignment 4}
\author{Your name and email}
\date{}
\maketitle

\section*{Task 1 (5 pts): Conjugate Function 1}
Define 
\begin{align}
  \Delta_n \doteq \qty{ x \in \R^n \mid \sum_{i=1}^n x_i = 1, x_i > 0}.
\end{align}
% Define $f: \R^n \to (-\infty, +\infty]$ as
Define $f: \Delta_n \to \R$ as
\begin{align}
  f(x) \doteq \sum_{i=1}^n x_i \ln x_i.
\end{align}
Then please compute the conjugate $f^*(y)$ of $f(x)$.
\begin{proof}
  \begin{align}
    f^*(y) =& \max_{x \in \Delta_n}\qty{y^\top x - f(x)}.
  \end{align}
  Consider the Lagrangian
  \begin{align}
    L(x, \lambda) = y^\top x - \sum_{i=1}^n x_i \ln x_i - \lambda (\sum x_i - 1).
  \end{align}
  Let $(x_*, \lambda_*)$ satisfy
  \begin{align}
    \nabla_\lambda L(x_*, \lambda_*) =& 0, \\
    \nabla_x L(x_*, \lambda_*) =& 0.
  \end{align}
  Then $x_*$ is the maximizer of the constrained optimization problem.
  \sz{Please complete the rest.}
\end{proof}


\section*{Task 2 (5 pts): Conjugate Function 2}
Define a function $f: \R^n \to \R$ as
\begin{align}
  f(x) = \ln\sum_{i=1}^n \exp(x_i).
\end{align}
Please compute the conjugate $f^*(y)$ of $f(x)$.

\begin{proof}
  \sz{Please compete the computation.}
\end{proof}


\section*{Task 3 (5 pts): Smoothness}
Let $\norm{\cdot}$ be the $\ell_2$ norm.
Let $C$ be a convex and closed set.
Recall that
\begin{align}
  P_C(x) \doteq& \arg\min_{y\in C} \norm{y - x}, \\ 
  d_C(x) \doteq& \min_{y \in C} \norm{y - x}, \\
  \psi_C(x) \doteq& \frac{1}{2} d_C^2(x).
\end{align}
Please prove that $\psi_C(x)$ is 1-smooth in $\norm{\cdot}$.
Hints:
\begin{enumerate}
  \item You can use the analytical expression of $\nabla \psi_C(x)$ directly. We computed this in a lecture. This is meant to reward those actually attending lectures.
  \item You can use the fact that $P_C(x)$ is firmly nonexpansive directly. We never discussed this property in lecutre and this is meant to practice your skill of ``proof with Wikipedia'' (cf. ``coding with StackOverflow'').
\end{enumerate}
\begin{proof}
  \sz{Please complete the proof.}
\end{proof}



\end{document}