\documentclass[11pt]{article}

% Any additional packages needed should be included after jmlr2e.
% Note that jmlr2e.sty includes epsfig, amssymb, natbib and graphicx,
% and defines many common macros, such as 'proof' and 'example'.
%
% It also sets the bibliographystyle to plainnat; for more information on
% natbib citation styles, see the natbib documentation, a copy of which
% is archived at http://www.jmlr.org/format/natbib.pdf

% \usepackage[preprint]{jmlr2e}
\usepackage[margin=1in]{geometry}
\usepackage{amssymb}
\usepackage{natbib}
\usepackage{color}
% \usepackage{float}
% \usepackage{hyperref}
\usepackage[hidelinks,colorlinks=true,citecolor=blue]{hyperref}
\usepackage{amsfonts}       
\usepackage{url}
\usepackage{amsmath}
\usepackage{amssymb}
\usepackage{amsthm}
\usepackage{wrapfig}
\usepackage{mathtools}
\usepackage{tikz}
\usepackage{subfig}
\usepackage{algorithmic}
\usepackage{subfig}
\usepackage{footmisc}
\usepackage{bibentry}
\usepackage{mathtools}
\DeclarePairedDelimiter{\ceil}{\lceil}{\rceil}
\usepackage{physics}
\usepackage{mathrsfs}
\mathtoolsset{showonlyrefs}
\usepackage{etoolbox}
\usepackage{makecell}
\usepackage[ruled]{algorithm2e}
\DontPrintSemicolon
\usepackage{enumerate}
\usepackage{booktabs}
\usepackage{listings}

\newcommand{\sz}[1]{({\color{blue} {SZ: #1}})}
\newcommand{\rl}[1]{({\color{green} {RL: #1}})}
\newcommand{\rt}[1]{({\color{red} {RT: #1}})}
\newcommand{\fS}{\mathcal{S}}
\newcommand{\fA}{\mathcal{A}}
\newcommand{\fY}{\mathcal{Y}}
\newcommand{\fX}{\mathcal{X}}
\newcommand{\fN}{\mathcal{N}}
\newcommand{\mY}{\mathbb{Y}}
\newcommand{\mP}{\mathbb{P}}
\newcommand{\fB}{\mathcal{B}}
\newcommand{\mI}{\mathbb{I}}
\newcommand{\fZ}{\mathcal{Z}}
\newcommand{\mZ}{\mathbb{Z}}
\newcommand{\fW}{\mathcal{W}}
\newcommand{\fF}{\mathcal{F}}
\newcommand{\fO}{\mathcal{O}}
\newcommand{\fH}{\mathcal{H}}
\newcommand{\mH}{\mathbb{H}}
\newcommand{\fU}{\mathcal{U}}
\newcommand{\R}{\mathbb{R}}
\newcommand{\Q}{\mathbb{Q}}
\newcommand{\E}{\mathbb{E}}
\newcommand{\nsa}{{|\fS \times \fA|}}
\newcommand{\ns}{{|\fS|}}
\newcommand{\na}{{|\fA|}}
\newcommand{\ny}{{|\fY|}}
\newcommand{\tb}[1]{{\textbf{#1}}}
\DeclarePairedDelimiter\floor{\lfloor}{\rfloor}

\begin{document}

\title{CS 4501: Optimization - Assignment 5}
\author{Your name and email}
\date{}
\maketitle

\part{}

The goal of this part is to implement mirror descent with $w: \R^n_+ \to \R$ defined as
\begin{align}
  w(x) = \sum_{i=1}^n x_i \ln x_i,
\end{align}
where $\R_+ \doteq (0, \infty)$

\section*{Task 1 (1pt): Compute $\nabla w(x)$}
  \sz{Complete the computation}
\begin{align}
  \pdv{w(x)}{x_i} = 
\end{align}

\section*{Task 2 (2pts): Compute $w^*(y)$}
\begin{align}
  w^*(y) \doteq \max_{x \in \R^n_+} y^\top x - w(x).
\end{align}
  \sz{Complete the computation}

\section*{Task 3 (1pt): Compute $\nabla w^*(y)$}
  \sz{Complete the computation}
\begin{align}
  \pdv{w^*(y)}{y_i} = 
\end{align}

\section*{Task 4 (2pt): Implement Mirror Descent w.r.t. $w$}
Define 
\begin{align}
  f(x) = \norm{x}^2_2.
\end{align}
Then implement
\begin{align}
  x_{k+1} = \nabla w^*\left(\nabla w(x_k) + \alpha \nabla f(x_k)\right).
\end{align}

\part{}

The goal of this part is to implement mirror descent with $w: \Delta_n \to \R$ defined as
\begin{align}
  w(x) = \sum_{i=1}^n x_i \ln x_i,
\end{align}
where 
\begin{align}
  \Delta_n \doteq \qty{ x \in \R^n \mid \sum_{i=1}^n x_i = 1, x_i > 0}.
\end{align}

\section*{Task 5 (2pt): Compute $\nabla w(x)$ and $\nabla w^*(y)$}
\sz{You can use results from Assignment 4 directly.}
\begin{proof}
  \sz{Complete the computation}
\end{proof}


\section*{Task 6 (2pt): Implement Mirror Descent w.r.t. $w$}
Define 
\begin{align}
  f(x) = \norm{x}^2_2.
\end{align}
Then implement
\begin{align}
  x_{k+1} = \nabla w^*\left(\nabla w(x_k) + \alpha \nabla f(x_k)\right).
\end{align}

\end{document}