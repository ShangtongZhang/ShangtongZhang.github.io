\documentclass[11pt]{article}

% Any additional packages needed should be included after jmlr2e.
% Note that jmlr2e.sty includes epsfig, amssymb, natbib and graphicx,
% and defines many common macros, such as 'proof' and 'example'.
%
% It also sets the bibliographystyle to plainnat; for more information on
% natbib citation styles, see the natbib documentation, a copy of which
% is archived at http://www.jmlr.org/format/natbib.pdf

% \usepackage[preprint]{jmlr2e}
\usepackage[margin=1in]{geometry}
\usepackage{amssymb}
\usepackage{natbib}
\usepackage{color}
% \usepackage{float}
% \usepackage{hyperref}
\usepackage[hidelinks,colorlinks=true,citecolor=blue]{hyperref}
\usepackage{amsfonts}       
\usepackage{url}
\usepackage{amsmath}
\usepackage{amssymb}
\usepackage{amsthm}
\usepackage{wrapfig}
\usepackage{mathtools}
\usepackage{tikz}
\usepackage{subfig}
\usepackage{algorithmic}
\usepackage{subfig}
\usepackage{footmisc}
\usepackage{bibentry}
\usepackage{mathtools}
\DeclarePairedDelimiter{\ceil}{\lceil}{\rceil}
\usepackage{physics}
\usepackage{mathrsfs}
\mathtoolsset{showonlyrefs}
\usepackage{etoolbox}
\usepackage{makecell}
\usepackage[ruled]{algorithm2e}
\DontPrintSemicolon
\usepackage{enumerate}
\usepackage{booktabs}
\usepackage{listings}

\newcommand{\sz}[1]{({\color{blue} {SZ: #1}})}
\newcommand{\rl}[1]{({\color{green} {RL: #1}})}
\newcommand{\rt}[1]{({\color{red} {RT: #1}})}
\newcommand{\fS}{\mathcal{S}}
\newcommand{\fA}{\mathcal{A}}
\newcommand{\fY}{\mathcal{Y}}
\newcommand{\fX}{\mathcal{X}}
\newcommand{\fN}{\mathcal{N}}
\newcommand{\mY}{\mathbb{Y}}
\newcommand{\mP}{\mathbb{P}}
\newcommand{\fB}{\mathcal{B}}
\newcommand{\mI}{\mathbb{I}}
\newcommand{\fZ}{\mathcal{Z}}
\newcommand{\mZ}{\mathbb{Z}}
\newcommand{\fW}{\mathcal{W}}
\newcommand{\fF}{\mathcal{F}}
\newcommand{\fO}{\mathcal{O}}
\newcommand{\fH}{\mathcal{H}}
\newcommand{\mH}{\mathbb{H}}
\newcommand{\fU}{\mathcal{U}}
\newcommand{\R}{\mathbb{R}}
\newcommand{\Q}{\mathbb{Q}}
\newcommand{\E}{\mathbb{E}}
\newcommand{\nsa}{{|\fS \times \fA|}}
\newcommand{\ns}{{|\fS|}}
\newcommand{\na}{{|\fA|}}
\newcommand{\ny}{{|\fY|}}
\newcommand{\tb}[1]{{\textbf{#1}}}
\DeclarePairedDelimiter\floor{\lfloor}{\rfloor}

\begin{document}

\title{CS 4501: Optimization - Assignment 3}
\author{Your name and email}
\date{}
\maketitle

Use Projected Gradient Descent (PGD) to minimize $f(x)$ starting from a given $x_0$ for $n$ iterations.
\begin{align}
  x_{n+1} = P_C\left(x_n - \alpha_n \nabla f(x_n)\right).
\end{align}

\section*{Task 1}
\paragraph*{Projection to a centered ball $(x \in \R^N, r > 0)$}
\begin{align}
  C = \qty{ x | \norm{x}_2 \leq r}
\end{align}
\begin{proof}
  If $\norm{x}_2 \leq r$, then objectively we have
  \begin{align}
    P_C(x) = x.
  \end{align}
  Now suppose $\norm{x}_2 > r$.
  Then we have $P_C(x) = y$, where $y$ is the solution for the following optimization problem.
  \begin{align}
    \min_y \quad & \norm{x - y}_2^2 \\
    \qq{subject to}& \norm{y}_2^2 \leq r^2
  \end{align}
  The Lagrangian of this optimization problem is
  \begin{align}
    L(y, \lambda) \doteq& \norm{x - y}^2_2 + \lambda(\norm{y}^2_2 - r^2)
  \end{align}
  where $x \in \R^N, y \in \R^N, \lambda \in \R$.
  Then we need to find $(y_*, \lambda_*)$ such that
  \begin{align}
    \nabla_y L(y_*, \lambda_*) =& 0, \\
    \nabla_\lambda L(y_*, \lambda_*) =& 0.
  \end{align}
  Then this $y_*$ is the solution to the optimization problem,
  i.e.,
  \begin{align}
    P_C(x) = \begin{cases}
      x, &\norm{x}_2 \leq r, \\
      y_*, &\norm{x}_2 > r
    \end{cases}.
  \end{align}
  \sz{Please compute $y_*$ here}
\end{proof}

\section*{Task 2}
\paragraph*{Projection to a noncentered ball $(x \in \R^N, c \in \R^N, r > 0)$}
\begin{align}
  C = \qty{ x | \norm{x - c}_2 \leq r}.
\end{align}
We have
\begin{align}
  P_C(x) = \begin{cases}
    x, & \qq{$\norm{x-c}_2 \leq r$} \\
    c + \frac{r}{\norm{x - c}_2}(x - c), &\qq{otherwise}
  \end{cases}.
\end{align}
\sz{No proof is required.}

\section*{Task 3}
\paragraph*{Projection to column space ($A \in \R^{N \times M}$ has full column rank but not necessarily square)}
\begin{align}
  C = \qty{ x | \exists y \in \R^M \qq{such that} Ay = x}.
\end{align}
\begin{proof}
\sz{Please compute the analytical expression of $P_C(x)$ here}
\end{proof}

\section*{Notes}
\begin{enumerate}
  \item Third-party packages, excluding numpy, are not allowed.
  \item The proof of Task 1 is 4 points. The proof of Task 3 is 2 points. The points for implementation are documented in the python script.
\end{enumerate}


\end{document}